\documentclass[a4paper,12pt,oneside]{book}

\usepackage{minted}

\usepackage{mlmodern}

\usepackage[utf8]{vietnam}
\usepackage{caption}
\newcommand{\source}[1]{\caption*{Source: {#1}} }

\usepackage{lipsum}
\usepackage{tikz}
\setlength{\parskip}{1em}
\usetikzlibrary{positioning, fit, calc}   
\tikzset{block/.style={draw, thick, text width=4cm ,minimum height=1.3cm, align=center},line/.style={-latex}}
\usepackage{indentfirst}          
% Dùng để set margin
\usepackage[top=3cm,inner=3.5cm,outer=2cm,bottom=3cm,headheight=120pt]{geometry}		

\usepackage{graphicx}			% Dùng để chèn hình ảnh
\graphicspath{ {images/} }		% Folder chứa hình ảnh
\usepackage{subfiles}			% Dùng để chia thành các files nhỏ
\usepackage{fancyhdr}			% Dùng để tạo header và footer
\usepackage{emptypage}			% Không đánh số trang, header, footer với các trang trắng
\usepackage{tabularx}			% Dùng để chèn bảng
\usepackage{pdflscape}			% Một vài trang nằm ngang (landscape) thay vì dọc (portrait) (bảng, biểu đồ)
\usepackage[square]{natbib}							% Dùng cho tài liệu tham khảo
\setcitestyle{super}								% Ký hiệu tài liệu tham khảo dạng superscript
\usepackage{url}									% Dùng cho URL
\usepackage{longtable}								% Dùng cho các bảng dài, nhiều trang
\usepackage{amsmath}								% Dùng để chèn các công thức Toán
\usepackage{makecell}								% Dùng để làm nổi bật header của bảng
\usepackage{float}								% Dùng để thiết đặt thuộc tính H (here) cho ảnh

% Định dạng cho header của bảng
\renewcommand\theadfont{\bfseries}
\makeatletter\@addtoreset{chapter}{part}\makeatother%

% Định dạng cho code
\usepackage{fancyvrb}			% Phiên bản nâng cao của verbatim, dùng để chèn code
\fvset{
	tabsize=4,					% Định tabsize 
	frame=single,				% Khung
}

% Hiển thị dòng kẻ để viết tay (phần Nhận xét của giảng viên)
\usepackage{pgffor, ifthen}
\newcommand{\notes}[3][\empty]{
	\noindent \vspace{15pt}\\
	\foreach \n in {1,...,#2}{
		\ifthenelse{
			\equal{#1}{\empty}
		}
		{\rule{#3}{0.5pt}\vspace{15pt}\\}
		{\rule{#3}{0.5pt}\vspace{15pt}\\}
	}
}

\title{Tên đề tài}						% Tiêu đề luận văn
\author{Họ tên sinh viên}				% Người thực hiện 

% Header và footer 
\pagestyle{fancy}
\fancyhf{}

% Header và footer nếu in 1 mặt
\rhead{\fontsize{10}{12} \selectfont Giảng viên hướng dẫn:\\TS.Trần Công Án}					% Header bên phải
\lhead{\fontsize{10}{12} \selectfont Đề tài:\\ Xây dựng Layer 2 cho Ethereum dùng phương pháp zk-Rollup}									% Header bên trái
\rfoot{\fontsize{10}{12} \selectfont \thepage}											% Footer bên phải
\lfoot{\fontsize{10}{12} \selectfont Võ Thành Vũ - B1606954}							% Footer bên trái

\begin{document}
\subfile{bia-ngoai}		% Gọi file bìa ngoài
\subfile{bia-trong}		% Gọi file bìa trong

\pagenumbering{roman}	% Kiểu số trang: i, ii, iii, iv, v,...

\chapter*{Nhận xét của giảng viên}
\subfile{chapters/nhan-xet-gv}

\chapter*{Lời cảm ơn}
\subfile{chapters/loi-cam-on}

\tableofcontents
\listoffigures
\listoftables

\clearpage

\begin{centering}
	\section*{Tóm tắt nội dung}
\end{centering}

Ethereum càng phổ biến và có nhiều ứng dụng được xây dựng trên nền tảng Ethereum. Tuy nhiên số lượng TPS có thể xử lý trong Ethereum là rất thấp làm ảnh hưởng rất tới trải nghiệm người dùng, gây lãng phí về thời gian và tiền bạc. Bài toán scalability (mở rộng) trong Ethereum trở thành bài toán cấp thiết cần giải quyết. Trong luận văn này tôi cố gắn xây dựng một layer 2 mở rộng từ Ethereum để giải quyết vấn đề scalability cho ethereum.



\begin{centering}
	\section*{Abstract}
\end{centering}

Ethereum is a blockchain platform very popular and there are many applications built on the Ethereum platform. However, the limit of the TPS makes process transaction time in Ethereum is very low, greatly affecting the user experience, wasting time and money. The scalability problem in Ethereum becomes an urgent problem to be solved. In this thesis, I tried to build a layer 2 extension from Ethereum to solve the scalability problem for Ethereum.
\pagenumbering{arabic}			% Kiểu số trang: 1, 2, 3,...

\part{Giới thiệu}
\chapter{Giới thiệu}
\subfile{chapters/chapter01}	% Gọi file chương 1, tương tự gọi các chương khác

\part{Nội Dung}
\chapter{Cơ sở lý thuyết}
\subfile{chapters/chapter02}

\chapter{Phương pháp thực hiện}

\subfile{chapters/chapter03}
\part{Kết Luận}
\chapter{Kết luận và phương hướng phát triển}
\subfile{chapters/chapter04}
\bibliography{mybib}{}
\bibliographystyle{siam}
\end{document}
